%----------------------------------------------------------------------------
\chapter{Bevezető}
%----------------------------------------------------------------------------

% Gyakorlás szerepe a tanulásban
A tanulás egy nehéz és időigényes folyamat, melynek megértésével és tökéletesítésével napjainkban is aktívan foglakozik a tudomány.
Ennek a folyamatnak elengedhetetlen része az önállóan végzett feladatmegoldás, a gyakorlás, mely során az elméleti tudás a gyakorlatban is hasznosítható tapasztalattá válik.
Gyakorlás közben fontos, hogy minél gyakrabban -- ha lehet, folyamatosan -- ellenőrizzük munkánk eredményét, ezzel biztosítva azt, hogy ne keletkezzenek rossz berögződések, melyek utólagos kijavítása további időt igényelne.

Az ellenőrzés módozatait többféle nézőpontból is értékelhetjük.
Az egyik ilyen nézőpont az ellenőrzési módszer flexibilitása, amely azt mutatja meg, mennyire változtatható meg a feladat úgy, hogy az ellenőrzési módszer továbbra is alkalmazható maradjon.
A spektrum egyik végén állnak az inflexibilis módszerek, amilyen például egy adott feladathoz mellékelt megoldókulcs.
A legegyszerűbb megoldókulcsok egyedül a feladat elvárt eredményét adják meg, ezért ezek flexibilitása igen csekély, hiszen a feladatban bármilyen érdemi változtatást elvégezve elvesztik érvényességüket. 
Ugyanakkor ezek a leggyorsabb és legolcsóbb módszerek, hiszen egyetlen összehasonlítás szükséges a saját és a megoldókulcs által megadott érték között, a megoldókulcsok pedig minimális költséggel sokszorosíthatóak és felhasználhatóak.
A spektrum másik végén találhatóak az oktatók -- általános esetben az adott terület szakértői --, akik a feladat témakörében való jártasságuk révén gyakorlatilag bármely megoldás ellenőrzésére képesek.
Az oktató azonban drága erőforrás, mind időben, mind pénzben, hiszen egy adott téma szakértőjeként a tanulók feladatainak ellenőrzése helyett az iparban is nagy szükség lenne rá.
Ebből fakadóan -- sok oktató alkalmazása költséges, ugyanakkor az iparnak nagyszámú szakértőre van szüksége -- az egy oktatóra eső hallgatók száma igen magas, amely körülmény nem kedvez az oktatás színvonalának, hiszen így minden hallgatóra kevesebb ideje jut az oktatónak.
Ha a spektrum két végének előnyeit -- könnyű sokszorosíthatóság, gyors működés, de magasabb szintű flexibilitás szakértelem hozzáadásával -- összehozzuk a számítástechnika által nyújtott automatizálási lehetőségekkel, eljuthatunk az \textit{automatizált feladatkiértékelő rendszerek} ötletéhez.


\section{Automatizált feladatkiértékelés}

\section{Online oktatást segítő rendszerek}
% TODO: Feladat indokoltsága: miért/mire jók az online oktatási rendszerek, a feladatbeadó rendszerek, automatikus kiértékelés
%       Eddigi megoldások (Coursera, Khan Academy, edX, Cporta)[http://www.lifehack.org/articles/money/25-killer-sites-for-free-online-education.html], hallgatói és oktatói szemszög


\section{A diplomaterv felépítése}
A dolgozatot az elvégzett munka és továbbfejlesztési lehetőségek összefoglalásával zárom.
