%----------------------------------------------------------------------------
\chapter*{Összefoglalás}\addcontentsline{toc}{chapter}{Összefoglalás}
%----------------------------------------------------------------------------

Az oktatói munka intenzitásának növekedése és a számonkérések minősége iránti elvárás növekedése az automatizálás általános előretörésével összhangban napjainkra elérte az egyetemi hallgatók önálló feladatainak kiadását és ellenőrzését is.
Ez az Irányítástechnikai és Informatikai Tanszéken, a bevált, de a kor növekvő igényeinek már nem mindenben megfelelő Cporta automatikus feladatkiadó és -értékelő rendszer szolgáltatásainak kibővítését igényelte.
Ezen elvárásokra válaszként több záródolgozat született.
Az én feladatom, ezekhez csatlakozva a Cporta utódjaként megjelenő Jporta feladatkiértékelő képességeinek gazdagítása volt.

A Jporta feladatkiértékelő funkciójának fejlesztése igen komplikált feladat, de véleményem szerint sikerült eredményeket elérnem.
A dolgozat \ref{chapter:jporta}. fejezetében bemutattam a Cporta tapasztalataira építő Jporta célját, használatát és történelmi előzményeit.
A \ref{chapter:exercise}. fejezetben leírtam a Jporta automatizált feladatkiértékelő moduljának továbbfejlesztését.
Ebben a követelmények felméréséből kiindulva kidolgoztam a feladatkiértékelő modul működéséhez szükséges adatszerkezeteket és szolgáltatási funkciókat.
Ezek a feladatok kezelésének négy periódusa köré csoportosultak.
Az első funkció a feladatok leírása, az oktatók általi, Jporta segítségével történő elkészítése.
A második az adatbázisban tárolt feladatok oktatók általi kiadása, a harmadik pedig a megoldások hallgatók általi beadása.
A negyedik a legizgalmasabb feladat, a megoldások ellenőrzése és kiértékelése.

A feladatleíró modulhoz kidolgoztam a modellosztályokat, melyek kapcsolatrendszerét egy UML osztálydiagramon keresztül ismertettem.

A feladatok készítését végző modulnál a feladatokra vonatkozó minél kisebb megszorítások alkalmazását, azaz a leírható feladatok sokszínűségét helyeztem előtérbe.
Az alkalmazott interfész, bár vizualizációban nem nyújtja a maximumot, de a feladatelemeket és azok összeköttetéseit kellő szinten megjelenítve nyújt felhasználóbarát megoldást.
Ehhez a modulhoz szorosan kapcsolódik az elkészült feladatok kiadását megvalósító funkció.

A hallgatók által használt feladatbeadás funkció alapvetően fájlok feltöltésének kezelését jelentette ismétlési és ellenőrző megtekintési szolgáltatások beépítésével.

A negyedik, megoldások ellenőrzése és értékelése funkció igényelte a legtöbb munkát és különféle szoftvertechnológiák (Celery, RabbitMQ, Redis) alkalmazását.
A Celery fejlett üzenetküldési képességeit használtam ki a beadott feladatok várakozási sorának kezelésére, melynél az üzenetszétosztást a RabbitMQ AMQP alapú bróker végzi.
A Redist, mely a kiértékeléskor kapott produktumok tárolásában működik közre, kiváló gyorsítótár tulajdonsága, és átfogó utasításkészlete miatt választottam.
Ehhez az összetett feladatkiértékelő rendszerhez több, a feladatok összeállításánál félkész megoldásként szolgáló blokkot készítettem el és ismertettem.
Ezek egyike a teljesen új lehetőséget nyújtó LecturerApproval blokk, mely az automatikus értékelést opcionálisan kiegészítő oktatói kézi értékelést is lehetővé teszi.

A \ref{chapter:features}. fejezetben további hasznos funkciókra tettem javaslatot, ezek a csapatrendszer és a verziókövető rendszerrel való integrálás.

A csapatrendszer integrálja a Jportába a tanszéken lévő Hercules nevű rendszer csapatkészítő és menedzsment funkcióját, ugyanakkor az eddig Herculest használó tárgyak számára nyújtja a Jporta széleskörű szolgáltatásait.
A megvalósításhoz megadtam a Jporta csapatmunka kezeléssel integrált modelljét, és a funkció kidolgozásához szükséges alapvető meggondolásokat.

A Jporta verziókövetéssel való integrálásának előnye abban áll, hogy hallgatók feladatmegoldás közben gyakorolhatják a verziókövető rendszerek használatát, megismerkedhetnek működésükkel és előnyeikkel.
Javaslatom részeként kiválasztottam egy kurrens verziókezelő szoftvert, a Gitet, és fő vonalakban bemutattam, hogyan képezelhető el feladatok beadása ezzel a rendszerrel.

\section*{További fejlesztési lehetőségek}\addcontentsline{toc}{section}{További fejlesztési lehetőségek}
A Jporta fejlesztése közel sem ért véget, a rendszerben még rengeteg potenciál rejtőzik.

A feladatok összeállításához felhasznáható blokkok száma és funkcionalitása még korlátozott.
Szükség lenne további blokkok tervezésére és készítésére, hogy a rendszer több programozási nyelvet, több tesztelési opciót is támogathasson.

A feladatok készítésére szolgáló felület ugyan kiválóan használható, de az általam is említett, vizuális, csomópont alapú szerkesztővel még hatékonyabbá és intuitívabbá válna ez a munkafolyamat.

A röviden megemlített végrehajtó modulok fejlesztése is gyerekcipőben jár még.
A fordítók, szkriptek, tesztek és egyéb programok biztonságos és hatékony futtatásának módszere még kidolgozásra vár.

A Jporta rengeteg izgalmas és kihívásokkal teli problémát tartogat még a következő generáció számára is.