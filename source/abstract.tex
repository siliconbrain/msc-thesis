%----------------------------------------------------------------------------
% Abstract in hungarian
%----------------------------------------------------------------------------
\chapter*{Kivonat}%\addcontentsline{toc}{chapter}{Kivonat}

A tanulási folyamatban jelentős szerep jut az önálló gyakorlásnak, feladatmegoldásnak, amelyhez fontos, hogy eredményességéről minél gyorsabb és megbízhatóbb visszajelzést kapjon a tanuló.
Ezt nagyban segítheti, ha a szükséges ellenőrzések minél nagyobb része automatizáltan történik, sok munkát levéve ezzel az oktatók válláról.
A programozás oktatása ebből a szempontból kitüntetett helyzetben van, hiszen a programok helyességének, elvárt működéstől való eltéréseinek vizsgálata már szinte a kezdetektől magas fokú automatizálással történik.

Az utóbbi időben, az online oktatás lehetőségeinek felismerésével, a szélessávú internethozzáférések mindenki számára elérhetővé válásával elterjedtté váltak az oktatást segítő platformok, melyek a hallgatók és az oktatók munkáját a tantermeken kívül is segítik.
Ilyen rendszer a BME Irányítástechnika és Informatika Tanszékén működő Jporta is, amely a programozás oktatását automatikus feladatkiértékelést és ellenőrzést végző funkciója mellett egyéb adminisztrációs oktatói feladatok támogatásával igyekszik segíteni.

Diplomatervemben bemutatom a Jporta rendszer működését, és a már említett feladatkiértékelő alrendszer tervezésének lépéseit.
Ez utóbbi részeként ismertetem a modullal szemben támasztott követelményeket, és ez alapján megtervezek egy, az automatizáltan kiértékelhető feladatok leírására alkalmas struktúrát, és az ezzel elkészített feladatokra érkező, hallgatók által készített megoldások kiértékelésére képes rendszert.
Végül javaslatot teszek két új funkció megvalósítására, melyeket röviden részletezek.
\vfill

%----------------------------------------------------------------------------
% Abstract in english
%----------------------------------------------------------------------------
\begin{otherlanguage}{english}
\chapter*{Abstract}%\addcontentsline{toc}{chapter}{Abstract}

There is a considerable part in the process of learning that goes to individual practice and problem solving where getting fast and reliable feedback about their progress is essential for students.
This can benefit greatly from doing substancial parts of the necessary verification automatically, freeing educators from significant work.
From this perspective the education of programming is in a special position because validation and testing of programs was performed with high levels of automation from the beginning.

In recent times, with the recognition of potential in online education and the widespread availability of broadband internet connectivity we see the proliferation of platforms supporting education that aid the work of students and teachers outside of the classroom.
Jporta found at the Department of Control Engineering and Information Technology at BUTE is such a system providing assistance for educators with administrative tasks beside its automatic submisson evaluation and testing facility.

In my master's thesis I'll introduce the operation of Jporta and the design process of the aformetioned submission evaluation subsystem.
As part of this, I'll present the requirements for the module, than I'll design a structure suitable for the description of automatically evaluable assignments and a system capable of evaluating submissions made by students for assignments created via said structure.
Lastly, I'll suggest two new features which I'll detail briefly.
\end{otherlanguage}
\vfill

