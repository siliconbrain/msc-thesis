%----------------------------------------------------------------------------
\chapter{Automatizált feladatkiértékelő modul}\label{chapter:exercise}
%----------------------------------------------------------------------------

\section{Követelmények}
% Igényfelmérés menete (a feladatkiértékelő modulról)
% Igények, elvárások, követelmények

\section{A feladatkiértékelő modul működése}
% Feladatkiértékelés nagyvonalakban (architektúrális nézet, named pipes & filters)

\section{Feladatok leírása}
% Feladatleíró struktúra

\section{Feladatok készítése}
% Feladatkészítés ebben a rendszerben (jelenlegi vs. vizuális nyelv)

\section{Feladatok kiadása}
% Feladatkiadás

\section{Feladatbeadás}
% Feladatbeadás

\section{Megoldások ellenőrzése és értékelése}
% Feladatkiértékelés konkrétan (Celery, AMQP/RabbitMQ, xattr (portal fs vs. backup fs támogatás), redis)

% Beágyazás a JPORTÁba (Django, felület) (ez inkább Viktoré?)
% Dokumentáció (a jelen dolgozat mellett...)

% Fejlesztés menete: git, Vim, CIRCLE devenv
